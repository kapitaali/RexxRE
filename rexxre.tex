%% Documentation for RexxRe

%  The contents of this file are subject to the Mozilla Public License
%  Version 1.0 (the "License"); you may not use this file except in
%  compliance with the License. You may obtain a copy of the License at
%  http://www.mozilla.org/MPL/

%  Software distributed under the License is distributed on an "AS IS"
%  basis, WITHOUT WARRANTY OF ANY KIND, either express or implied. See the
%  License for the specific language governing rights and limitations
%  under the License.

%  The Original Code is RexxRE.

%  The Initial Developer of the Original Code is Patrick TJ McPhee.
%  Portions created by Patrick McPhee are Copyright � 2003
%  Patrick TJ McPhee. All Rights Reserved.

%  Contributors:

% $Header: C:/ptjm/rexx/rexxre/RCS/rexxre.tex 1.2 2003/05/26 18:47:13 ptjm Rel $

\documentclass{article}

% if run through pdftex, use ps fonts and turn on hyperlinking
\ifx\pdfoutput\undefined
 \def\rarrow{\ensuremath{\to}}
 \def\hyperlink#1#2{}
 \def\href#1#2{}
\else
 \usepackage[pdftex,pdfborder=0 0 0]{hyperref}
 % uncomment for older hyperref (or update your hyperref!)
 % \def\pdfBorderAttrs{/Border [0 0 0]}%
 \def\_{\textunderscore\penalty\hyphenpenalty}
 \usepackage[T1]{fontenc}
 \usepackage{times}
 \usepackage{mathptm}
 \pdfinfo { /Title (Rexx Regular Expression Library)
  /Author (Patrick TJ McPhee)
  /Keywords (rexx,regina,regular expressions)
  /Version (1.0.1)
  /Copyright (Copyright 2003, Patrick TJ McPhee)
 }
\fi

\makeatletter
\def\verbatim@font{\normalfont\ttfamily\small}
\makeatother

\begin{document}
\let\thepage\relax
\setcounter{page}{0}

\title{Rexx Regular Expression Library}
\author{Patrick TJ McPhee (ptjm@interlog.com)}
\date{Version 1.0.1, 26 May 2003}

\maketitle

\eject

\pagenumbering{roman}

\tableofcontents

\vfil\eject

\pagenumbering{arabic}

\section{Introduction}

This paper describes RexxRE, a set of functions which provide regular
expression matching as defined in the POSIX specification. Regular expressions
are commonly used in data parsing applications, and provide a useful extension
to the Rexx function set, although some Rexx programmers consider the
datatype(\,) function to be an adequate replacement.

A regular expression (RE) is a string which represents some set of strings.
The
point of the regular expression routines is to search some target string for
strings which can be represented by a regular expression.
The simplest RE is a literal representation of exactly one string, for
instance the RE `and' represents the string `and'. More usefully, special
characters can be used to broaden the set of strings represented by the RE.
The RE `roo?t' represents the strings `root' and `rot', while `[a-f0-9]\char`\{2,10\char`\}'
represents the set of hexadecimal numbers from 2 to 10 digits long. Section
\ref{sec:regularexpressons} gives a complete description of the regular
expression syntax.

The POSIX regular expression interface defines four functions to
perform regular expression matching. regcomp(\,) converts a
string representation of the regular expression into a format
which can be efficiently compared to input strings. regexec(\,)
performs this comparison. regerror(\,) returns error information in
a human-readable format. regfree(\,) releases the memory used to
hold a compiled regular expression. Each of these functions has
a direct analog in RexxRE.
RexxRE includes an additional function which attempts to
emulate the popular parse instruction using regular expressions.

On Unix systems, the regular expression routines are provided by
the system. On Win32 and OS/2 platforms, I use the well-known regular
expression routines by Henry Spencer of the University of
Toronto, which are expected to have been brought into POSIX
compliance by members of the CSRG at UC Berkley and by the
FreeBSD project.

\subsection{Installation}

There is no installation program.

On Win32 platforms, copy win32/rexxre.dll to a directory on your
path, or the directory containing regina.exe or your rexx-enabled
application. If you use a Rexx interpreter other than Regina,
you either need to recompile the library using your interpreter's
development kit or download the rexx/trans version of RexxRE.

On OS/2, copy os2/rexxre.dll to a directory in your LIBPATH.

On Unix, you need to compile the library. 
The distribution does not include a configuration script, but it includes
make files which have been known to work using the stock vendor compiler
on several Unix systems. If you have one of those systems, link the
appropriate make file to the name `Makefile' and build the `dist'
target. For instance, on Solaris:
\begin{verbatim}
ln Makefile.sun Makefile
make dist
\end{verbatim}

On most platforms, this builds a shared library called librexxre.so.
On HP-UX, the file is called librexxre.sl, and on AIX, it's called
librexxre.a. The path to this library can be set in three ways:

Most Unix systems allow a shared library search path to be embedded
into program files. If you build regina (or your rexx-enabled
application) such that this path is set to include a directory
such as /opt/regina/lib or /usr/local/lib, you can install
RexxRE by copying the shared library to this directory. Otherwise,
you need to either set an environment variable or change the way
the system searches for shared libraries.

Unix systems typically use a different path for shared libraries than
they do for program files. The name of the environment variable used
for the shared library path is not standardised, however most systems
use LD\_LIBRARY\_PATH. Notable exceptions are AIX (LIBPATH) and
HP-UX (SHLIB\_PATH for 32-bit executables, LD\_LIBRARY\_PATH for
64-bit executables). To install RexxRE, add an appropriate directory
to the shared library path for your machine and copy the shared
library to that directory.

Finally, some systems provide a utility (often called ldconfig) which
can be used either to set the standard search path for shared libraries,
or to provide a database of shared libraries. On such a system, RexxRE
can be installed by copying the shared library to an appropriate
directory and using this utility to add it to the search database.
You'll need to consult your system documentation for more information.

\subsection{Reporting Bugs}

I don't anticipate making a lot of changes to this library in the
future, but I would like it to be bug-free.

If you find a bug, an error in the documentation, or you simply have
a suggestion for improving the distribution, please send me details 
at ptjm@interlog.com. It's useful to know the operating system you're
using, the version of Regina (or Rexx/IMC), and the version of rexxre,
and to have a set of steps for reproducing the bug.

\subsection{Using RxFuncAdd}

All the routines in rexxre can be loaded either directly using
RxFuncAdd\index{RxFuncAdd}, or indirectly using ReLoadFuncs\index{ReLoadFuncs}.
RxFuncAdd takes three arguments~-- the name of the function as it will
be used in the rexx program, the name of the library from which to load
the function, and the name of the function as it appears in the library.

RxFuncAdd returns 0 on success, or 1 on failure. Regina has a function
called RxFuncErrMsg which can give useful information about the reason
for a load failure. A few common reasons for failure are:

Path issues: the library is called rexxre.dll on Win32 and OS/2 platforms,
librexxre.a on AIX, librexxre.sl on HP-UX, and librexxre.so on
other Unix platforms. On Win32, this file needs to be in the path, or
in the directory containing regina.exe. On OS/2, it needs to be
in a directory specified in LIBPATH. On Unix systems, it needs
to be in a directory listed in LIBPATH on AIX, SHLIB\_PATH on HP-UX
32-bit, or LD\_LIBRARY\_PATH on most other Unix systems. Some systems
have an ldconfig utility which allows you to forego setting this environment
variable.

Windows 95: early releases of windows 95 did not include msvcrt.dll, the
C run-time library used by RexxRE. This library is sometimes installed
with applications software. It can also be obtained through service packs,
or from the Microsoft web site.

Rexx.exe: Regina includes two executables, one called `rexx', and
the other called `regina'. The difference is that `rexx' includes the Rexx
interpreter as part of the executable, while `regina' loads the interpreter
from a shared library. RxFuncAdd works only with the `regina' version of
the interpreter (the `rexx' version is slightly faster, though).

\subsection{Licencing}

RexxRE is distributed free of charge in the hopes that it will be
useful, but without any warranty. It is distributed under the terms of
the Mozilla Public License. The precise details of the licence are found
in the file MPL-1.1.txt in the distribution. Henry Spencer's regular
expression routines are taken from the FreeBSD distribution and are
distributed under the BSD licence.

\goodbreak

\section{Regular Expressions}

\label{sec:regularexpressons} This section describes the regular expression
(RE) format defined by the POSIX standard. Because the regular expression
processing is actually handled by operating-system-specific functions, this
syntax might not exactly match the syntax available on your system, and
there's little I can do about it. Most current systems should work as
described here. For others, you might try compiling with the Spencer routines
found in the regex sub-directory.

For historical reasons, there are two regular expression syntaxes defined by
POSIX, which are usually called `basic' and `extended'. We are encouraged to
use `extended' REs, however most Unix utilities actually use `basic' REs, and
the default format for the regular expression routines is `basic'. To make
things more confusing, many Unix utilities extend either type of RE in one way
or another, many editors provide their own unique regular expression syntax,
the regular expression library included with Object Rexx is similar to, but
different from, `extended' regular expressions, and some people are familiar
with the regular expressions provided by perl, which are syntactically
completely different from anything else anywhere. I have elected to stick with
the POSIX definitions since they are the only reasonable standard.

\subsection{Common Syntax}

Both `basic' and `extended' regular expressions consist of strings of
characters, some of which have special meanings. The regular expression is expressed in terms of `ordinary'
characters, which correspond to themselves, and `meta' characters,
which generally change the way the regular expression matching is performed.

The set of ordinary and meta characters differs between `basic' and
`extended' regular expressions, however there is some common ground.
In general, non-punctuation characters are always ordinary. `cat'
matches only the string `cat', and `12' matches only `12'.

The characters below are meta characters in both types of RE.
Later sections list the meta characters which are specific to each type
of RE.

\begin{enumerate}
\item[.]Matches any character. `c.t' matches `cat', `cot', `cut', `ctt',
{\it etc}.;
\item[[]Introduces an alternation set. `c[auo]t' matches `cat', `cut',
or `cot', but not `ctt' or any of the other things matched by `c.t'.
`\char`\^', `-', and `]' can have special meanings in alternation
sets. Alternation is discussed more fully in section \ref{sec:alternation};
\item[*]Kleene closure: matches 0 or more of whatever the preceding character
or group of characters matches. `.*' matches any string. `a*' matches
`', `a', `aa', {\it etc}.. The regular expression engine will take
the first, longest match for Kleene closure. `a*' will match the
first three characters of `aaabcaa';
\item[\textbackslash]Escapes itself and the other meta characters.
\textbackslash. matches `.' only. In `basic' regular expressions,
\textbackslash{} also causes some ordinary characters to act like
meta characters;
\item[\char`\^]When placed at the beginning of a regular expression,
anchors the regular expression to the start of the string being
searched. The regular expression `at' matches the last two characters of
`cat', but `\char`\^at' does not. Elsewhere in the regular expression,
\char`\^\ has no special meaning;
\item[\$]When placed at the end of a regular expression, anchors the
regular expression to the end of the string being searched. The regular
expression `ca' matches the first two characters of `cat', but `ca\$'
does not.
\end{enumerate}

\subsection{Basic Regular Expressions}

`Basic' regular expressions use \textbackslash{} to turn some ordinary
characters into meta characters.

\begin{enumerate}
\item[\textbackslash(]Begins a group. The regular expression which
appears between \textbackslash( and \textbackslash) is treated
as a unit with respect to Kleene closure and interval operations.
`\textbackslash(abc\textbackslash)*' matches `', `abc', `abcabc',
{\it etc}.;
Group contents can also be returned to the caller by the regular
expression matching functions, and can be used in back-references;
\item[\textbackslash)]Ends a group;
\item[\textbackslash{\it d}]Back-reference: matches the contents
of the {\it d}th group, where {\it d} is an integer from 1 to 9.
`\textbackslash([cd]\textbackslash)at\textbackslash1' matches
`catcat' and `datdat', but not `datcat';
\item[\textbackslash\char`\{]Interval: \textbackslash\char`\{ must
be followed by one of a whole number $w_1$, a whole number $w_2$ and a comma,
or two whole numbers $w_3$ and $w_4$, $w_3 <w_4$ separated by a comma.
The interval constrains the preceding character
or group to match exactly $w_1$ instances, $w_2$ or more
instances, or between $w_3$ and $w_4$ instances, inclusive, depending on
the syntax used. `a\textbackslash\char`\{3\textbackslash\char`\}' matches
the first three characters of `aaaaaa'.
`a\textbackslash\char`\{3,\textbackslash\char`\}' matches
all the characters of `aaaaaa' but does not match `aa'.
`a\textbackslash\char`\{3,4\textbackslash\char`\}' matches
the first four characters of `aaaaaa' but does not match `aa'.
\item[\textbackslash\char`\}]Closes an interval.
\end{enumerate}

\subsection{Extended Regular Expressions}

`Extended' regular expressions introduce a few new meta characters
which make certain kinds of regular expression more convenient to
type. They eliminate back-references, which makes other kinds of
regular expression impossible.

\begin{enumerate}
\item[+]Kleene closure. matches 1 or more of whatever the preceding character
or group of characters matches. `.+' matches any string of non-zero length.
`a+' matches `a', `aa', {\it etc}.. The regular expression engine will take
the first, longest match for Kleene closure. `a+' will match the
first three characters of `aaabcaa';
\item[?]Matches 0 or 1 of whatever the preceding character or group of
characters matches. `\char`\^noodles?\$' matches `noodle' or `noodles';
\item[(]Begins a group. The regular expression which
appears between ( and ) is treated
as a unit with respect to Kleene closure and interval operations.
`(abc)*' matches `', `abc', `abcabc',
{\it etc}.;
Group contents can also be returned to the caller by the regular
expression matching functions;
\item[)]Ends a group;
\item[|]Branching: divides the regular expression (or group within
a regular expression) into alternatives. The regular expression will
match the first, longest match of any of the alternatives. `cat|dog'
will match either `cat' or `dog'. It will match the first three
characters of `cats and dogs'. `\char`\^\ *(int|long) +[a-z\textunderscore][a-z0-9A-Z\textunderscore]*;'
matches the declaration of an int or long variable in a C program;
\item[\char`\{]Interval: \char`\{ must
be followed by one of a whole number $w_1$, a whole number $w_2$ and a comma,
or two whole numbers $w_3$ and $w_4$, $w_3<w_4$ separated by a comma.
The interval constrains the preceding character
or group to match exactly $w_1$ instances, $w_2$ or more
instances, or between $w_3$ and $w_4$ instances, inclusive, depending on
the syntax used. `a\char`\{3\char`\}' matches
the first three characters of `aaaaaa'.
`a\char`\{3,\char`\}' matches
all the characters of `aaaaaa' but does not match `aa'.
`a\char`\{3,4\char`\}' matches
the first four characters of `aaaaaa' but does not match `aa'. Note that
Object Rexx supports only the first interval syntax;
\item[\char`\}]Closes an interval.
\end{enumerate}

\subsection{Alternation Sets}

\label{sec:alternation}
Alternation is a useful but seemingly confusing aspect of regular
expression syntax. At its simplest, a set of characters delimited by
[ and ] matches any character in the set. [abc] matches any of
a, b, or c. The alternation set is treated as a single unit for
the purposes of Kleene closure and interval operations.

Alternation sets have a different set of meta characters from the
rest of the RE. For instance, `.' is an ordinary character within
an alternation set, but `\char`\^', `-', and `]' have special meanings.

\begin{enumerate}
\item[\char`\^]When it appears at the very start of the alternation set,
`\char`\^' inverts the set. `[\char`\^abc]' matches any character {\it but}
a, b, or c. When it appears elsewhere in the set, it has no special
meaning;
\item[-]When it appears between two other characters, - represents all
the characters which appear between those two characters in the current
character set. On ASCII systems, `[a-z]' represents any lower-case
letter. - represents itself when placed at either the start or the end
of the alternation set. `[\char`\^-]' represents any character but -. `[-+/*]'
represents any arithmetic operator;
\item[{]}]When it appears anywhere but at the start of the alternation
set, ] terminates the set. `[\char`\^]]' represents any character but ].
`[]a-z]' represents any lower-case letter or ].
\end{enumerate}

Since the use of character values and the use of the range operator
results in definitions which are character-set- and
natural-language-specific, POSIX defines classes of characters which can
be used by name within alternation sets, by enclosing the names within
`[:' and `:]'. These classes are:

\begin{enumerate}
\item[alnum]letters and numbers;
\item[alpha]letters;
\item[blank]space or tab;
\item[cntrl]control characters;
\item[digit]numbers;
\item[graph]printable characters;
\item[lower]lower-case letters;
\item[print]printable characters;
\item[punct]punctuation;
\item[space]white space (including newlines, vertical tabs, {\it etc.});
\item[upper]upper-case letters;
\item[xdigit]hexadecimal digits.
\end{enumerate}

A C language identifier consists of a letter or underscore followed by
any number of letters, digits, or underscores. This could be matched
by the regular expression `[a-zA-Z\textunderscore][a-zA-Z0-9\textunderscore]*', however this will
work only on machines that use the ASCII character set. The same
regular expression using character classes is
`[[:alpha:]\textunderscore][[:alnum:]\textunderscore]*'.

Note that the POSIX treatment of character classes is different from
the Object Rexx treatment.

\section{Function Definitions}

\subsection{Introduction}

Normal use of RexxRE is to register the library using ReLoadFuncs(\,),
compile one or more regular expressions using ReComp(\,), then match
the compiled regular expressions against input strings using ReExec(\,).
Alternatively, ReParse(\,) can be used to perform an operation similar
to the parse value instruction, using regular expressions for the parse
template.

Compiling regular expressions is optional, but you should
note that there is no way to determine whether the regular expression
was compiled correctly by either ReExec(\,) or ReParse(\,), and
compiling regular expressions provides a performance gain unless
the regular expression match is performed only once.

If an error does occur while compiling the regular expression, useful
diagnostic may be retrieved by calling ReError(\,). Once the compiled
regular expression is no longer needed, it can be released using
ReFree(\,).



\subsection{ReLoadFuncs}

\begin{verbatim}
ReLoadFuncs()
\end{verbatim}

ReLoadFuncs(\,) registers all the other routines in RexxRE with the Rexx
interpreter. This registration takes less work on your part than
registering each individual function using RxFuncAdd(\,), but you can
register the functions you intend to use individually if that makes you feel
better.

\subsection{ReDropFuncs}

\begin{verbatim}
ReDropFuncs()
\end{verbatim}

ReDropFuncs(\,) unregisters all the routines in RexxRE. It is safe
to do this even if the functions were not registered using
ReLoadFuncs(\,). With Regina and likely most other interpreters, function
registration ends when the Rexx process terminates in any case,
however with IBM's interpreters, function registration persists until
the functions are unregistered or the machine is rebooted, so
it's good practice to call ReDropFuncs(\,) at the end of each program.

\subsection{ReVersion}

\begin{verbatim}
ReVersion() -> version string
\end{verbatim}

ReVersion(\,) reports the version number of the library in the format
{\it version}.\penalty\exhyphenpenalty{\it release}.\penalty\exhyphenpenalty{\it modification}. It's often
useful to have this information when investigating bugs.

Typically,
I change the {\it modification} level whenever I make a bug-fix
release, I change the {\it release} number whenver I add new
functions, and I change the {\it version} number when I add
dramatically new functionality. This might happen if I elected
to support the Object Rexx regular expression syntax, for instance.

\subsection{ReComp}

\begin{verbatim}
ReComp(re, flags) -> cre
\end{verbatim}

\label{sec:recomp}
ReComp(\,) compiles the regular expression {\it re} into the compiled
regular expression {\it cre}. By default, {\it re} is a `basic'
regular expression, and it is compiled to match case-insensitively,
meaning lower-case letters in {\it re} will match the corresponding upper-
and lower-case letters.

You need to call ReComp(\,) only once per regular expression, and
you can have any number of compiled regular expressions active at any time.
Both ReExec(\,) and ReParse(\,) accept uncompiled regular expressions as
well as compiled regular expressions, so you may leave out the
compilation step if you don't need to perform repeated matches
against the same RE.

You should release the memory allocated for the regular expression by
calling ReFree(\,) when you no longer need it.

{\it Flags} can be any combination of

\begin{enumerate}
\item[x]means {\it re} is an `extended' regular expression;
\item[c]to get case-sensitive matches
\item[s]means that only the status of the match will be returned from
ReExec(\,). When this flag is given, ReExec(\,) will return as soon as some
match of the regular expression has been achieved.
Normally the regular expression engine will continue processing
until it finds the first, longest match, and it returns either the
matched strings or their position. Do not pass the {\it matches} argument
to ReExec(\,) if you pass the `s' flag to ReComp(\,), and do not
use a compiled regular expression which used the `s' flag as an argument
to ReParse(\,);
\item[n]Treat new-lines as match delimiters. Normally, a new-line in
a target string is treated like any other control character, but
when the regular expression is compiled with the `n' flag, new-lines
are never matched by the regular expression and are used to delimit
RE anchors ({\it e.g.}, `cat\$' would match `cat[newline]').
\end{enumerate}

If the regular expression compiles successfully, the first character of
{\it cre} will be 0. Otherwise, it will be 1. You can test it like
this:
\begin{verbatim}
cre = ReComp('cat$', 'c')
if left(cre, 1) then do
  say 'failed to compile RE cat$:' ReError(cre)
  call ReFree cre
  exit 1
  end 
\end{verbatim}

{\samepage

\subsection{ReExec}

\begin{verbatim}
ReExec(cre, string[, matches[, flags]]) -> 0 or 1
\end{verbatim}

ReExec(\,) searches the target string {\it string} for the first,
longest substring which can be represented by the regular expression {\it cre}.
{\it cre} ought usually to have been compiled by a previous call to ReComp,
however it can also be a string representation of the RE, at the cost
of performance.

}

{\it matches} is the name of a stem used to store information about
the match. On successful return, {\it matches}.0 contains the number
of parenthesized sub-expressions of {\it cre}.
Each of {\it matches}.1 to {\it matches}.({\it matches}.0)
contains the string that matches the corresponding parenthesised
subexpression of {\it cre}. {\it matches}.!match  contains the string which
matches the whole of {\it cre}. Do not pass a value for {\it matches}
if {\it cre} was compiled with the `s' flag.

{\it flags} can be any combination of the flag arguments to ReComp(\,)
and these values:
\begin{enumerate}
\item[p]instead of returning strings in {\it matches}, return the
offset and length of each match within {\it string}, separated by
a single space;
\item[b]the string does not start at the beginning of a line. This
affects \char`\^\ processing;
\item[e]the string does not end at the end of a line. This affects
\$ processing.
\end{enumerate}

ReComp(\,) flags are used only when compiling a string-format regular
expression. I don't advise doing this as it's usually faster
to use the result of ReComp(\,), and there's no way to test for
errors in regular expression compilation when it's done by ReExec(\,).
In any case, passing, for instance `c' as a flag to ReExec(\,) will
have no effect if {\it cre} is a compiled regular expression.

ReExec returns 1 if a match was found for the regular expression, and
0 otherwise.

\subsection{ReError}

\begin{verbatim}
ReError(cre) -> error string
\end{verbatim}

ReError(\,) takes a compiled regular expression returned from a failed call
to ReComp(\,) and returns a string which explains the problem.

\subsection{ReFree}

\begin{verbatim}
ReFree(cre[, cre, ...])
\end{verbatim}

ReFree(\,) releases memory used to hold the compiled regular expressions {\it
cre}. For complicated regular expressions, memory usage can be substantial,
and it makes sense to call ReFree(\,) as soon as you no longer need the RE.

\subsection{ReParse}

\begin{verbatim}
ReParse(cre, string, [flags], varname[, varname, ...]) -> 0 or 1
\end{verbatim}

ReParse(\,) attempts to provide `parse value'-like functionality using regular
expressions. The arguments are a regular expression, which should normally
have been compiled by ReComp(\,), a target string, flags that affect the
processing of the regular expression, and a list of variable names. The
regular expression is used to divide the target string into fields, which are
then assigned to the variables as described below.

By default, {\it cre} represents a field delimiter, and each field is assigned
to the corresponding variable. As with the parse instruction, if more data is
available than there are variable names, all the data at the end of {\it
string} is written to the final variable in the list. These lines should be
equivalent
\begin{verbatim}
call ReParse '  *', value, , a, b, c
parse value value with a b c
\end{verbatim}

{\it Flags} can be any combination of the flag arguments to ReComp(\,)
and these values, which determine how fields are assigned to variables:

\begin{enumerate}

\item[d]{\it cre} represents the delimiter between each field. If the
delimiter appears at the start of {\it string}, then the first field has zero
length. If the number of variables is less than then number of delimited
fields, the last variable in the list is assigned the remaining portion of the
string, excluding the leading delimiter. If the number of variables is greater
than the the number of delimited fields, all the unmatched variables are set
to zero-length. This is the default;

\item[v]{\it cre} represents the format of a field. If each field consists of
integer data, {\it cre} might be `[0-9]+'. If $n$ variable names are passed as
arguments, up to the first $n - 1$ field values are assigned to the
corresponding variables from the list. The $n$th variable is always assigned
the portion of {\it string} which remains after the field assignments. It can
then be used for further parse operations. If there are fewer than $n$ fields,
any variables which do not correspond to a field are set to zero length;

\item[s]each field corresponds to a parenthesised sub-expressions. If the
number of variables is less than the number of sub-expressions, the extra
matches will be thrown away. Note that this is essentially the opposite
meaning to the `s' flag of ReComp(\,);

\item[t]{\it varname} is the name of a stem variable, and fields are written
to the stem using the numeric convention.

\end{enumerate}

The ReComp(\,) flags are used only if {\it cre} is a string representation
of the RE.
You should compile any regular expression which will be used in the
parse or exec routines more than once, since the performance is better.
You should also compile any regular expression which is created
dynamically or input by the end user, in order to validate its
compilation.

\subsubsection{Examples}

Parsing a syslog entry:

\begin{verbatim}
  /* here's a syslog entry */
  data = 'Apr 29 11:55:17  su: ptjm to root on /dev/ttyp2'

  /* this RE matches some characters which are not :, a space,
   * two digits forming sub-expression 1, a colon, two digits
   * forming sub-expression 2, a colon, two digits forming
   * sub-exression 3, any number of spaces, some bunch of
   * non-colon characters forming sub-expression 4, a colon, a space,
   * and whatever's left in the target string, forming sub-expression 5 */
  re = '[^:]+ ([0-9]{2}):([0-9]{2}):([0-9]{2}) +([^:]+): (.+)'

  /* the parse call matches the sub-expressions to the variables
   * hh, mi, ss, component, and msg, respectively */
  call reParse re, data, 'xs', 'hh', 'mi', 'ss', 'component', 'msg'
  /* now, hh = 11; mi = 55; ss = 17; component = 'su'
          msg = 'ptjm to root on /dev/ttyp2' */

  /* this RE matches any number of spaces and colons */
  re = '[ :]{1,}'

  /* the parse call matches each space-delimited expression to a
   * variable */
  call reParse re, data, , ., ., 'hh', 'mi', 'ss', 'component', 'msg'
  /* now, hh = 11; mi = 55; ss = 17; component = 'su'
          msg = 'ptjm to root on /dev/ttyp2' */

  /* this RE matches any number of alpha-numerics */
  rean = ReComp('[[:alnum:]]+', 'x')

  /* this one matches any number of numerics */
  ren = ReComp('[[:digit:]]+', 'x')

  /* this one matches any number of non-spaces or colons */
  re!sc = ReComp('[^ :]+', 'x')

  /* throw away the date */
  call reParse(rean, data, 'v', '.',  '.', data)
  /* data = ' 11:55:17  su: ptjm to root on /dev/ttyp2' */

  /* get the time */
  call reParse(ren, data, 'v', 'hh', 'mi', 'ss', data)
  /* data = '  su: ptjm to root on /dev/ttyp2' */

  /* get the component */
  call reParse(re!sc, data, 'v', 'component', data)

  /* data = ': ptjm to root on /dev/ttyp2' */
  msg = substr(data, 2)
  /* now, hh = 11; mi = 55; ss = 17; component = 'su'
          msg = 'ptjm to root on /dev/ttyp2' */
  call ReFree ren, rean, re!sc
\end{verbatim}

Flexible date processing:

\begin{verbatim}
parsedate: procedure
mwre = ReComp('[[:alpha:]]+', 'x')
mdre = ReComp('[[:digit:]]{1,2}', 'x')
yre = ReComp('[[:digit:]]{4}', 'x')

parse arg date

pdate = date
rdate = 'baddate'

/* test for April 30 2003 */
if ReParse(mwre, pdate, 'v', 'month', 'pdate') then do
    if ReParse(mdre, pdate, 'v', 'day', 'pdate') then do
        if ReParse(yre, pdate, 'v', 'year', 'pdate') then
             rdate = year'/'mwords(month)'/'day
        end
    /* start over, trying 30 April 2003 */
    if rdate = 'baddate' then do
        pdate = date
        if ReParse(mdre, pdate, 'v', 'day', 'pdate') then do
            if ReParse(mwre, pdate, 'v', 'month', 'pdate') then do    
                if ReParse(yre, pdate, 'v', 'year', 'pdate') then
                     rdate = year'/'mwords(month)'/'right(day,2,'0')
                end
            end
        end
    end
else do
    /* test for 2003/04/30 */
    pdate = date
    if ReParse(yre, pdate, 'v', 'year', 'pdate') then do
        if ReParse(mdre, pdate, 'v', 'month', 'pdate') then do    
            if ReParse(mdre, pdate, 'v', 'day', 'pdate') then
                rdate = year'/'right(month,2,'0')'/'right(day,2,'0')
            end
        end

    /* test for 30/04/2003, then give up */
    if rdate = 'baddate' then do
        pdate = date
        if ReParse(mdre, pdate, 'v', 'day', 'pdate') then do
            if ReParse(mdre, pdate, 'v', 'month', 'pdate') then do    
                if ReParse(yre, pdate, 'v', 'year', 'pdate') then
                    rdate = year'/'right(month,2,'0')'/'right(day,2,'0')
                end
            end
        end
    end

call ReFree mwre, yre, mdre
return rdate
\end{verbatim}

\end{document}
